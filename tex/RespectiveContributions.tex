\documentclass[11pt]{article}
\usepackage[T1]{fontenc}
%^\usepackage[utf8]{inputenc}
\usepackage{amsmath,mathrsfs,bm}
\usepackage{color}
%\usepackage{cleveref}
\usepackage{mathtools}
\usepackage{graphicx}
\usepackage[font=scriptsize,labelfont=bf]{caption}

\usepackage{natbib}
\usepackage[superscript,biblabel]{cite}

%\usepackage{fullpage}
\usepackage[margin=0.6in]{geometry}

\pagenumbering{gobble}

\usepackage{fontspec}
\setmainfont{Arial}

% \linespread{1.05}
\newcommand{\jb}[1]{{\color{blue} (#1)} }
\newcommand{\gs}[1]{{\color{red} #1}}

\title{Respective Contributions}
\date{}

\begin{document}

\maketitle

I (Jeremy Berg) have been a member of Guy Sella's lab since October 2016, and we began exploring the possibility of working extensively together on the evolution and population genetics of complex disease around that time. I wrote the application, with input from GS.

My proposed research will be highly collaborative, and take full advantage of the differing skill sets and knowledge bases present among members of the Sella lab. I will be responsible for leading all aspects of the proposed work, but will also benefit from significant input from GS, and from two talented graduate students in the lab, Yuval Simons and Laura Hayward, who's insights from their work the evolution of quantitative traits under stabilizing and directional selection respectively will be valuable for certain parts of my proposed work. Finally, some small aspects of the project may be delegated to a Columbia undergraduate student in the Sella lab who I am mentoring. The extent to which this is done will depend on her interest and skill progression, and decisions in this regard will be made in consultation with GS.

I will take responsibility for writing up results, again with significant input from GS.

\end{document}