\documentclass[11pt]{article}
\usepackage[T1]{fontenc}
\usepackage[utf8]{inputenc}
\usepackage{amsmath,mathrsfs,bm}
\usepackage{color}
\usepackage{cleveref}
\usepackage{mathtools}
\usepackage{graphicx}
\usepackage{fullpage}
\usepackage{natbib}
\usepackage{fontspec}
\setmainfont{Arial}

\linespread{1.05}
\newcommand{\jb}[1]{{\color{blue} (#1)} }
\newcommand{\gs}[1]{{\color{red} (#1)}}

\usepackage[margin=0.7in]{geometry}
\pagenumbering{gobble}

\usepackage{fontspec}
\setmainfont{Arial}

\title{Facilities}
\date{}
\begin{document}

\maketitle

\paragraph{Scientific environment}

At Columbia University, members of the Sella groups benefit from the research and training environment provided by the department of Biological Sciences. In particular, Dr Sella has especially close ties with two leading human population geneticists, Molly Przeworski, a member of the same department, and Joe Pickrell, who is affiliated with Biological Sciences but based primarily at the nascent New York Genome Center. The Pickrell-Przeworski-Sella groups run joint weekly lab meetings, in which members of each group present in turn, as well as a weekly journal club on papers in human genetics and evolutionary biology. More broadly, Dr Sella's group benefits from a strong community in computational biology and genomics on campus, including Harmen Bussemaker in Biological Sciences and Itsik Pe’er in Computer Science. Moreover, the medical school at Columbia recently created a Precision Medicine department, and parallels similar efforts at Mount Sinai and the NYU Medical School in NYC. Dr. Sella also has a secondary appointment in the Committee for Computational Biology and Bioinformatics, which provides ties with experimental and computational groups in genomics and human genetics. Thus, I will have an unparalleled opportunity to interact with and present my work to a wide range of researchers with expertise pertinent to the proposed research.

\paragraph{Dry Laboratory Space}
Dr. Sella’s dry lab space of 1385 square feet is located in a newly renovated section on the 6th floor of the Fairchild building, and includes three 3-4 person offices (133-150 sq. ft each), as well as a lounge area, small kitchen, and conference room (shared with Molly Przeworski). The lab is adjacent to the one run by Molly Przeworski and across the hall from that of Harmen Bussemaker, forming a stimulating computational biology research corridor.

\paragraph{Computing}
The main resource (beyond pencil and paper) necessary for the proposed work is computing power. The Sella lab has access and dedicated space on the powerful computer cluster of the Center for Computational Biology and Bioinformatics (http://systemsbiology.columbia.edu/advanced-research-computing-services).The system is on the Top500 of supercomputers worldwide, with more than 6,000 CPU cores and 73,000 GPU cores, a maximum performance of 212 TFlops and 1.4PB of high-speed storage space. All of our data is replicated daily to a secondary location, periodically written to tape and sent to an offsite storage location.

\paragraph{Financial support}

As a recent recruit to Columbia University, Dr. Sella benefits from start-up funds as well as lab space renovated to his specifications, and is also currently funded by an NIH R01 grant. Seeing as the primary financial expediture associated with my proposed work is likely to be for computation power, I forsee no difficulties covering these costs. 



\end{document}