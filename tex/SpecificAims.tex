\documentclass[a4paper,10pt]{article}
\usepackage[T1]{fontenc}
\usepackage[utf8]{inputenc}
\usepackage{amsmath,mathrsfs,bm}
\usepackage{color}
\usepackage{cleveref}
\usepackage{mathtools}
\usepackage{graphicx}
\usepackage{fullpage}
\usepackage{natbib}
\newcommand{\jb}[1]{{\color{blue} (#1)} }
\newcommand{\gs}[1]{{\color{red} #1}}
\begin{document}

Our ability to leverage and apply data regarding the genetics of complex disease depends critically on the extent to which we understand how factors such as the genetic architecture, pleiotropy, and changes to the environment affect the evolution of disease risk and its distribution across the population.
GWAS has produced vast amounts of data informing on which positions in the genome are associated with variation in disease liability, but the answers to basic questions about what sort of genetic architectures are consistent with available evidence remain deeply contestsed, with models ranging from univariate liability threshold models to various forms of allelic or genetic heterogeneity retaining supporters.
The extent to which these models have been formalized varies from one to the next, with some existing almost entirely in verbal form.
With a few notable exceptations, surprisingly little is known concretely about the population genetic consquences of invoking a particular biological/genetic model of complex disease \jb{be careful here to make sure I don't come across as if no one has ever thought about all this before/appropriately reference what has been done}.
A quantitative and formalized
Working in collaboration with Dr Guy Sella, I will develop, formalize, and analyze an array of population genetic models of complex disease evolution. I will explore and analyze these mode

things people have claimed to be able to infer on the basis of some feature of GWAS:
- more protective mutations than risk mutations -> must have been an environmental shift toward increased disease risk which is responsible for selecting protective alleles up
-

statistical genetics studies which seek to determine something about relationship between allele frequency and heritability explained are trying to make qualitative inference about genetic architecture. want to make the quantitative
\end{document}
