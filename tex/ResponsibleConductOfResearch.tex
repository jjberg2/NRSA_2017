\documentclass[11pt]{article}
\usepackage[T1]{fontenc}
\usepackage[utf8]{inputenc}
\usepackage{amsmath,mathrsfs,bm}
\usepackage{color}
\usepackage{cleveref}
\usepackage{mathtools}
\usepackage{graphicx}
\usepackage{fullpage}
\usepackage{natbib}
\linespread{1.05}
\newcommand{\jb}[1]{{\color{blue} (#1)} }
\newcommand{\gs}[1]{{\color{red} (#1)}}
\pagenumbering{gobble}
\usepackage[margin=0.7in]{geometry}

\usepackage{fontspec}
\setmainfont{Arial}

\title{Responsible Conduct of Research}
\date{}
\begin{document}

\maketitle

As part of my instruction in the responsible conduct of research during this fellowship, I will register for and complete the Columbia University Medical Center's course in the Responsible Conduct of Research and Related Policy Issues (coursenumber G4010). This course is designed to meet NIH training grant requirements for the responsible conduct of research, and to provide necessary tools for building a solid foundation for research integrity. It consists of eleven lecture-discussion sections of 1 hour each, spread over an academic semester, covering the following topics
\begin{itemize}
  \itemsep0em 
\item the mentee-mentor relationship
\item authorship practices and scientific publications
\item research involving human participants/subjects
\item data acquisition, ownership, sharing, management, and reproducibility
\item the use of laboratory animals in scientific research
\item conflicts of interest
\item peer review
\item intellectual property
\item the role of scientists in society
\item and collaborative research and partnerships with industry
\end{itemize}

These sessions will be taught and facilitated by members of the faculty and administration with particular knowledge and expertise of the specific topics, and it is my understanding that researchers at all levels from a variety of fields at Columbia University participate in this program. Dr Sella and I have agreed to discuss issues that come up during these sessions and are relevant to our research, and in general Dr Sella and I will be having regular (at least weekly, probably more often) to discuss the progress of our research, and we have agreed that these meetings are an appropriate place for discussion of wider ranging issues, including our mentor-mentee relationship, and the broader impacts of our research in society.


\end{document}